\documentclass{article}

% If you're new to LaTeX, here's some short tutorials:
% https://www.overleaf.com/learn/latex/Learn_LaTeX_in_30_minutes
% https://en.wikibooks.org/wiki/LaTeX/Basics

% Formatting
\usepackage[utf8]{inputenc}
\usepackage[margin=1in]{geometry}
\usepackage[titletoc,title]{appendix}

% Math
% https://www.overleaf.com/learn/latex/Mathematical_expressions
% https://en.wikibooks.org/wiki/LaTeX/Mathematics
\usepackage{amsmath,amsfonts,amssymb,mathtools}

% Images
% https://www.overleaf.com/learn/latex/Inserting_Images
% https://en.wikibooks.org/wiki/LaTeX/Floats,_Figures_and_Captions
\usepackage{graphicx,float}

% Tables
% https://www.overleaf.com/learn/latex/Tables
% https://en.wikibooks.org/wiki/LaTeX/Tables

% Algorithms
% https://www.overleaf.com/learn/latex/algorithms
% https://en.wikibooks.org/wiki/LaTeX/Algorithms
\usepackage[ruled,vlined]{algorithm2e}
\usepackage{algorithmic}

% Code syntax highlighting
% https://www.overleaf.com/learn/latex/Code_Highlighting_with_minted
\usepackage{minted}
\usemintedstyle{borland}

% References
% https://www.overleaf.com/learn/latex/Bibliography_management_in_LaTeX
% https://en.wikibooks.org/wiki/LaTeX/Bibliography_Management
\usepackage{biblatex}
\addbibresource{references.bib}

\DeclareMathOperator{\taninv}{tan\,inverse}

% Title content
\title{ENPM 673 Homework 1}
\author{Arjun Srinivasan Ambalam,Praveen Menaka Sekar,Arun Kumar Dhandayuthabani}
\newline
\date{February 11, 2020}

\begin{document}

\maketitle




% Introduction and Overview
\section{Problem 1}


% Example Subsection
\subsection{Since camera sensor is square shaped and also a single focal length is given,the Field of View (FOV) in Horizontal and Vertical directions are same}
\begin{equation*}
FOV =2*\phi , \linebreak
\linebreak
\phi=\tan ^{-1} \frac{d}{2f}
\end{equation*}
 where d is camera sensor width (d=14 mm)
        , f is focal length (f=15 mm)
\begin{equation*}     
   \phi =\tan ^{ - 1}\frac{14}{2*15}\end{equation*}
                    
                   \hspace{6.5cm} \phi =  25.0168.
                  \newline
                  \hspace{6.5 cm} FOV = 2*25.0168 = 50.036 \end


% Example Subsubsection
\subsection{Minimum number of pixels occupying image}
Square shaped object with width 5cm (w) = 50 mm,
Distance of from the camera (D) = 20 m = 20000 mm.
Let size of the object image on camera sensor be 'x' mm
\begin{equation*}
 \frac{x}{2f}=\frac{w}{2D}
 \end{equation*}
\begin{flushleft}
After substitution and solving we get x=0.0375 mm,\\
5 MP resolution for sensor area 14\times14\hspace{.1cm}mm ^2 \\
Therefore for sensor area \hspace{.1cm}0.0375*0.0375, Minimum number of corresponding pixels are 35.87
  
\end{flushleft}

\section{Curve Fitting Methods}
We have discussed about Least Squares ,Total Least Squares and Least Squares with Regularization.From the results we see that             method is best fit for the problem in hand. 

\section{Homography in Computer Vision}

\begin{flushleft}
To compute SVD for Matrix A ,we need to calculate  eigen values and vectors of A$A^{\rm T}$ and $A^{\rm T}$A matrices \\

\begin{equation*}
A = 
\begin{bmatrix}
-x1 & -y1 & -1 & 0 & 0 & 0 & x1*xp1 & y1*xp1 & xp1 \\
 0 & 0 & 0 & -x1 & -y1 & -1 & x1*yp1 & y1*yp1 &yp1 \\
-x2 & -y2 & -1 & 0 & 0 & 0 & x2*xp2 & y2*xp2 & xp2 \\
 0 & 0 & 0 & -x2 & -y2 & -1 & x2*yp2 & y2*yp2 &yp2 \\
-x3 & -y3 & -1 & 0 & 0 & 0 & x3*xp3 & y3*xp3 & xp3 \\
 0 & 0 & 0 & -x3 & -y3 & -1 & x3*yp3 & y3*yp3 &yp3 \\
-x4 & -y4 & -1 & 0 & 0 & 0 & x4*xp4 & y4*xp4 & xp4 \\
 0 & 0 & 0 & -x4 & -y4 & -1 & x4*yp4 & y4*yp4 &yp4 
\end{bmatrix}
\end{equation*}



where,

\begin{center}
\begin{tabular}{ |c|c|c|c|c|c| } 
\hline
index & x & y & xp & yp \\
\hline
\multirow
1 & 5 & 5 & 100 & 100 \\ 
2 & 150 & 5 & 200 & 80 \\
3 & 150 & 150 & 220 & 80 \\
4 & 5 & 150 & 100 & 200 \\
\hline
\end{tabular}
\end{center}

Substituting the values, the A matrix becomes

\begin{equation*}
A = 
\begin{bmatrix}
-5 & -5 & -1 & 0 & 0 & 0 & 500 & 500 & 100 \\
 0 & 0 & 0 & -5 & -5 & -1 & 500 & 500 & 100 \\
-150 & -5 & -1 & 0 & 0 & 0 & 30000 & 1000 & 200 \\
 0 & 0 & 0 & -150 & -5 & -1 & 12000 & 400 & 80 \\
-150 & -150 & -1 & 0 & 0 & 0 & 33000 & 33000 & 220 \\
 0 & 0 & 0 & -150 & -150 & -1 & 12000 & 12000 & 80 \\
-5 & -150 & -1 & 0 & 0 & 0 & 500 & 15000 & 100 \\
 0 & 0 & 0 & -5 & -150 & -1 & 1000 & 30000 & 200 
\end{bmatrix}
\end{equation*}



A &= U\Sigma V^{T} \\
The columns of U (size:m\times m) are \ orthogonal \ eigenvectors \ of A$A^{\rm T}$ \\
The columns of V (size:n\times n) are  \ orthogonal \ eigenvectors \ of $A^{\rm T}$A \\
Eigenvalues \lamda 1 … \lambda r \ of \ A$A^{\rm T}$V are the eigenvalues of \\
\sigma i=\sqrt {\lambda i}\\
\sum_=diag(\sigma 1...\sigma r)


U=\begin{bmatrix}
    0.0118  &  0.0003 &   0.0516  & -0.4661  & -0.2603  & -0.0678  &  0.0108 &  -0.8411\\
    0.0118  &  0.0003 &   0.0872 &  -0.4594  & -0.2491  & -0.0886  & 0.7655  &  0.3542\\
    0.3587  &  0.6549 & -0.0135  & -0.4651  &  0.1701  &  0.2936  & -0.2784  &  0.1823\\
    0.1435  &  0.2620 &   0.4454 &   0.1361  & -0.5008 &  -0.5875 &  -0.2731 &   0.1529\\
    0.7750  &  0.0227 &  -0.4085  &  0.2849 &   0.0320 &  -0.2352 &   0.2627  & -0.1597\\
    0.2818  &  0.0082 &   0.6922  &  0.3159  &  0.0114  &  0.5019  &  0.2466 &  -0.1696\\
    0.1846  & -0.3168 &  -0.2485 &  -0.0347  & -0.6983 &   0.4673  & -0.2524  &  0.1816\\
    0.3693  & -0.6336  &  0.2889  & -0.3933  &  0.3189  & -0.1750 &  -0.2614  &  0.1526\\
\end{bmatrix}\\
\vspace{5mm} %5mm vertical space
V=\begin{bmatrix}
    0.0028  &  0.0031  & -0.2464 &  -0.1586  &  0.1752 &   0.1767 &  -0.9137  &  0.1203 &   0.0531\\
    0.0024  & -0.0013  & -0.3770 &   0.1766  & -0.6895 &   0.5903 &   0.0529  &  0.0022 &  -0.0049\\
    0.0000  &  0.0000  & -0.0024 &  -0.0037  & -0.0052 &   0.0075 &  -0.0660  & -0.7860 &   0.6146\\
    0.0011  &  0.0012  &  0.6612 &   0.3412  & -0.5017 &  -0.2325 &  -0.3721  &  0.0426 &   0.0177\\
    0.0016  & -0.0029  &  0.5743 &  -0.0710  &  0.3145 &   0.7499 &   0.0620  & -0.0046 &  -0.0039\\
    0.0000  & -0.0000  &  0.0058 &  -0.0022  & -0.0029 &  -0.0057 &   0.1225  &  0.6049 &   0.7868\\
   -0.6961  & -0.7180  & -0.0001 &  -0.0038  & -0.0025 &  -0.0002 &  -0.0044  &  0.0006 &   0.0002\\
   -0.7180  &  0.6961  &  0.0016 &  -0.0038  & -0.0025 &   0.0037 &   0.0006  & -0.0000 &  -0.0000\\
   -0.0062  &  0.0000  & -0.1735 &   0.9067  &  0.3783 &   0.0622 &  -0.0252  &  0.0025 &   0.0076\\

\end{bmatrix}
\vspace{5mm} %5mm vertical space

\sum_= \begin{bmatrix}
 \vspace{5mm} %5mm vertial space

      60215 &         0    &     0  &       0   &      0    &     0   &      0     &    0     &    0\\
         0  &     31825     &   0   &      0    &     0     &    0    &     0   &      0      &   0\\
         0   &      0       & 261   &      0    &     0     &    0  &       0    &     0      &   0\\
         0   &      0       &  0    &    186    &     0     &    0     &    0    &     0      &   0\\
         0   &      0       &  0    &     0     &   146     &    0      &   0    &     0      &   0\\
         0   &      0       &  0    &     0     &    0      &   61      &   0    &     0      &   0\\
         0   &      0       &  0    &     0     &    0      &   0       &   4    &     0      &   0\\
         0   &      0       &  0    &     0     &    0      &   0       &  0     &     1      &   0\\
\end{bmatrix}



\end{flushleft}


\end{document}
